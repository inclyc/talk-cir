\section{现状}

\begin{frame}[fragile]
    \frametitle{开发进度与流程 -- LLVM Incubator}

    当前所有的源代码都在 \texttt{llvm/clangir} 这个仓库中。上面有若干提交,不定
    期 rebase 到 \texttt{llvm/llvm-project} 的主线分支上。

    \vspace{2em}
    \begin{lstlisting}[caption=启用ClangIR目前需要的编译指令]
        cmake /* .. */ \
        -DLLVM_ENABLE_PROJECTS="clang;mlir;cir;"
    \end{lstlisting}


\end{frame}

\begin{frame}
    \frametitle{Clang CodeGen}

    Clang IR 目前没有被用于真正的 CodeGen 流程,而是只用于静态分析。

    开发者的计划是
    \begin{center}
        Clang AST $\rightarrow$ Clang IR (Multi-Level) $\rightarrow$ LLVM IR
    \end{center}

    用于 CodeGen 主要有如下一些挑战:

    \begin{enumerate}
        \item 复用 CodeGen 代码,保证 ABI 兼容
        \item 编译时间
        \item C/C++ 生态事实上是相当复杂的,CIR是否可以支持 Open\{MP,Acc,CL\} ?
    \end{enumerate}
\end{frame}
