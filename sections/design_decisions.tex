\section{Design Decision}

\begin{frame}
    \frametitle{Why MLIR?}

    MLIR 给出了一套非常完善的测试框架、Pass 管理,我们可以用 Pass 来实现关于
    CIR 的各个分析和 Lowering。

    \begin{figure}
        \centering
        \includegraphics[width=0.2\textwidth]{images/mlir-logo.png}
        \caption{MLIR Logo}
    \end{figure}

    事实上,FIR (Flang 使用的 IR) 就是通过 MLIR 构建的。

\end{frame}


\begin{frame}[fragile]
    \frametitle{高级语义}
    \begin{center}
        \begin{minipage}{0.5\textwidth}
            \begin{lstlisting}[caption=一个简单的生命周期分析例子]
int *may_explode() {
    int *p = nullptr;
    {
    int x = 0;
    p = &x;
    *p = 42;
    }
    *p = 42; // oops...
    ...
}
            \end{lstlisting}
        \end{minipage}
    \end{center}

\end{frame}

\begin{frame}[fragile]
    \frametitle{对应的 CIR}
    \begin{center}
        \begin{minipage}{1.0\textwidth}
            \begin{lstlisting}
func @may_explode() -> !cir.ptr<i32> {
%p_addr = cir.alloca !cir.ptr<i32>, cir.ptr <!cir.ptr<i32>>, ["p", cinit]
...
cir.scope {
    // int x = 0;
    %x_addr = cir.alloca i32, cir.ptr <i32>, ["x", cinit]
    // p = &x;
    cir.store %x_addr, %p_addr : !cir.ptr<i32>, cir.ptr <!cir.ptr<i32>>
    // *p = 42
    cir.store %forty_two, %x_addr : i32, cir.ptr <i32>
    %p = cir.load deref %p_addr : cir.ptr <!cir.ptr<i32>>, !cir.ptr<i32>
} // 'x' lifetime ends, 'p' is bad.
// *p = 42
%forty_two = cir.cst(42 : i32)
%dead_x_addr = cir.load deref %p_addr : cir.ptr <!cir.ptr<i32>>, !cir.ptr<i32>
// attempt to store 42 to the dead address
cir.store %forty_two, %dead_x_addr : i32, cir.ptr <i32>
}
            \end{lstlisting}
        \end{minipage}
    \end{center}

\end{frame}
