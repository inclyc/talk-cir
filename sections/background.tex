\section{ClangIR提出的背景}

\begin{frame}
    \frametitle{Clang 数据流分析诊断}

    Clang 的数据流相关诊断信息目前是在前端部分实现的。LLVM IR 更类似与一种三地
    址中间码,大量的信息会在 lower 到 LLVM IR 的路径中丢失。

    \begin{table}
        \centering
        \begin{tabular}{cc}
            \toprule
            已有的问题                             & 问题描述                                             \\
            \midrule
            \texttt{\_\_builtin\_constant\_p} & 与 LLVM 行为实际不一致\cite{aaronballman-constantp-2019} \\
            无穷递归                              & missing-warnings\cite{aaronballman-infrec-2019}  \\
            Lifetime                          & 难以实现                                             \\
            高层次的代码变换建议                        & 难以实现                                             \\
            \bottomrule
        \end{tabular}
        \caption{目前 Clang AST \& CFG 遇到的问题}
    \end{table}
    \label{tab:clang_diag}
\end{frame}

\begin{frame}[fragile]
    \frametitle{Infinite Recursive}
    \begin{center}
        \begin{minipage}{0.7\textwidth}
            \begin{lstlisting}[caption=无限递归诊断\cite{lyc-infrec-2019}]
void func1(int i) { /* no warning ?? */
    if (i || !i)
        func1(i);
}

void func2(int i) {
    if (i > 0 || i <= 0) /* tautological! */
       func2(i);
}
            \end{lstlisting}
        \end{minipage}
    \end{center}
\end{frame}
